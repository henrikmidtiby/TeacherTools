\documentclass{article}
\usepackage[utf8]{inputenc}
\usepackage{todonotes}
\usepackage{graphicx}
\usepackage[colorlinks, linkcolor=blue, citecolor=blue, urlcolor=blue]{hyperref}

\title{Latex Beamer Course exporter -- Freeplane export filters}
\author{Henrik Skov Midtiby}

\begin{document}
\maketitle

This document describes how the freeplane export filters 
\emph{Latex Beamer Course exporter (.tex)} and 
\emph{Latex Beamer Course outline exporter (.tex)} can be used.
The two export filters was developed by Henrik Skov Midtiby
during 2015 and 2016, for simplifying handling of lecture
notes for two classes.


The export filters has been based on mm\_xslt\_exports
export filters by Igor G. Olaizola
\footnote{\url{https://github.com/igor-go/mm_xslt_exports}}.

\section{Getting started}

\subsection{Installation}

\begin{enumerate}
\item	Download the export filters from my
\href{https://github.com/henrikmidtiby/TeacherTools/tree/master/freeplane%20export%20filters}{github repository}.
\item	Place the files (mm2courseexporter.xsl and mm2courseexporteroverview.xsl) in the \texttt{freeplane-1.3.15/resources/xslt} directory.
\item	Launch freeplane.
\end{enumerate}

\subsection{What is placed where}

In the mindmap that should be exported to beamer slides, 
the export filters makes the following assumptions about
the content of the different nodes, based on their distance 
to the root node.
\begin{description}
\item[level 0 (root node)] Course name	
\item[level 1]	Name of the lecture
\item[level 2]	Name of current topic
\item[level 3] 	Name of current subtopic
\item[level 4]	Actual slides 	
\end{description}

\section{Options }

\subsection{Level 0 -- Course name}

\begin{description}
\item[]	
\item[]	
\item[]	
\end{description}

\subsection{Level 1 -- Lecture name}

\begin{description}
\item[]	
\item[]	
\item[]	
\end{description}


\subsection{Level 2 -- Topic}

\begin{description}
\item[]	
\item[]	
\item[]	
\end{description}

\subsection{Level 3 -- Subtopic}

\begin{description}
\item[]	
\item[]	
\item[]	
\end{description}

\subsection{Level 4 -- Slide}

\begin{description}
\item[]	
\item[]	
\item[]	
\end{description}



\end{document}